% !TeX root = ../diss.tex

\chapter{Introduction}
The Smith-Waterman (SW) algorithm is used by bioinformaticians to find similarities between DNA or protein structures.
I implemented this algorithm using three different processor architectures: CPUs, GPUs, and FPGAs.
All of these architectures allow for parallel computation but lend themselves to extracting parallelism from the algorithm in different ways.
This dissertation discusses the relative strengths and weaknesses of these different architectures to implement this algorithm, and the varying approaches required for these different platforms.

\section{Motivation}
Sequence alignment involves finding the sections of two sequences that are most similar to one another, and recording how they differ.
This can be useful when comparing genes or proteins between species, or when identifying genetic variations or diseases between individuals of a species.
In the context of bioinformatics, sequences of amino acids make up proteins, which perform a multitude of biological functions, and sequences of nucleotides make up DNA which encodes for the construction of these proteins and controls their usage.

There is a vast quantity of sequenced protein and DNA data available, for example the GRC’s most recent human genome assembly contains over 3.1 billion base pairs \cite{GRCh38}.
Attempting to find patterns is simply intractable for humans, and sequence alignment algorithms are used instead to identify similar regions for further investigation.
This can be particularly helpful given a labelled database of sequences, such as the UniProt Knowledgebase \cite{UniProt}, a database of proteins.
Using such a database, new proteins can be aligned against database sequences to find the most similar proteins that have been previously studied, providing clues on the properties of the new protein.

\section{Previous work}
The Smith-Waterman algorithm \cite{SW_Original} is often used to align sequences, yet it is computationally expensive, requiring $O(NM)$ space and time to run in its original form, for sequences of lengths $N$ and $M$ (see \cref{sec:SW_Complexity}).
Its structure exposes a significant degree of parallelism, and there have been previous investigations into implementing it on GPUs and FPGAs for increased performance when compared to CPUs.
Investigating ways of accelerating this algorithm may lead to time and money being saved, though hardware availability is often the constraining factor in practice.
For this reason most alignments are performed on CPUs and GPUs, but it is academically interesting to explore producing a hardware accelerator nonetheless.

In this project I use and combine some previous approaches to produce implementations for each platform and compare them.
This was done several years ago by Benkrid et al. \cite{Benkrid12}, though I used more modern hardware.
This project repeats their experiment, and I investigated the impact of different scoring mechanisms on performance, which was not done by Benkrid et al.

% !TeX root = ../diss.tex

\chapter{Conclusions}

In conclusion, this project was successful.
The core criteria, producing single and multi-threaded C, and CUDA implementations of the Smith-Waterman algorithm, were achieved.
As was one of the extensions, to implement and instantiate the algorithm on an FPGA.
Investigations into the impact of different ways of scoring alignments were performed, and the performance of the different implementations were compared.

\section{Results}
\label{sec:Concl_results}

As I initially suspected, the Smith-Waterman algorithm was amenable to parallelisation, with the GPU implementation outperforming the CPU implementation for large sequences.
However, smaller alignments have less scope for parallelisation, and the CPU could outperform the GPU with better sequential performance.

I only had time to make an implementation for the FPGA that could align short sequences.
The accelerator I designed was highly specialised and was able to do a lot of work each clock cycle, albeit at a modest clock speed.
For these short sequences, it outperformed the CPU and GPU implementations by a significant margin.

\section{Further work}
\label{sec:Concl_further work}

I cannot think of any major insufficiencies in the C and CUDA implementations I made.
Undoubtedly improvements could be made by hand-optimising the assembly code, but I suspect this would not yield major improvements.
An extension which I did not have time for was to modify my C implementation to use the Streaming SIMD Extensions of the x86 platform to improve the performance of my CPU-based implementation, and this could be investigated.

However, the FPGA implementation was ambitious and has obvious areas for improvement.
Alone it could have been the basis for a whole Part II Project, but using good planning I used the time I had to produce a limited design that could align sequences.
Two areas of improvement are:
\begin{itemize}
\item Modifying the design to align much longer sequences, using the linear space approach (\cref{sec:SW_Linear_Prep}).
This would require the original, quadratic space implementation to perform the alignments of sufficiently small subproblems.
Either a larger FPGA could be used where both designs could be instantiated together, or perhaps they could be combined where the accelerator chooses its alignment algorithm based on sequence length.

\item Improving the clock speed of the design through pipelining.
The critical paths in my design were inside of processing elements, performing long sequences of arithmetic and comparisons each clock cycle.
This arithmetic could be divided into pipeline stages, allowing the clock to run at a higher frequency.

\end{itemize}
